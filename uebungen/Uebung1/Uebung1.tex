\documentclass[a4paper]{scrartcl}
\usepackage{scrpage2}
\usepackage[ngerman]{babel}
\usepackage[T1]{fontenc}
\usepackage[utf8]{inputenc}
%\usepackage[pdftex]{graphicx}
%\usepackage[intlimits]{amsmath}
%\usepackage{listings}
%\lstset{frame=single,breaklines=true}
\usepackage{ amssymb }
\usepackage{amsmath}
\usepackage{hyperref}
\usepackage{enumerate}
\pagestyle{scrheadings}
\pagenumbering{gobble}
\ihead{Übungsblatt 1\\Nils Werner 108012219293}
\chead{\\Paul Rösler 108012225686	}
\ohead{Übungsgruppe: Mo. 16:00\\Daniel Teuchert 108012214552}
\setheadsepline{0.4pt}
\begin{document}

\section*{Aufgabe 1}
\begin{enumerate}[a)]

%a)
\item Für die Basiszustände $|\spadesuit\rangle$, $|\bigstar\rangle$, $|\blacklozenge\rangle$ gilt:\\
$\langle\spadesuit|\spadesuit\rangle = 1$, $\langle\bigstar|\bigstar\rangle = 1$, $\langle\blacklozenge|\blacklozenge\rangle = 1$\\\\

$\langle x|x\rangle= \frac{1}{2}\cdot\frac{1}{2}\cdot\langle\spadesuit|\spadesuit\rangle + (-\frac{1}{2})\cdot(-\frac{1}{2})\cdot\langle\bigstar|\bigstar\rangle+\frac{1+i}{2}\cdot\frac{1-i}{2}\cdot\langle\blacklozenge|\blacklozenge\rangle= \frac{1}{4}+\frac{1}{4}+\frac{1-i^2}{4}= \frac{1}{2}+\frac{1}{2}=1$\\
$\langle y|y\rangle= \frac{i}{2}\cdot(\frac{-i}{2})+(\frac{-i}{2})\frac{i}{2}+\frac{1-i}{2}\cdot\frac{1+i}{2}=\frac{1}{4}+\frac{1}{4}+\frac{1}{2}=1$\\
$\langle w|w\rangle= \frac{1}{3}\cdot\frac{1}{3}+(\frac{-2i}{3})\cdot(\frac{2i}{3})+(\frac{-2i}{3})\cdot(\frac{2i}{3})=\frac{1}{9}+\frac{4}{9}+\frac{4}{9}=1$\\\\

$\langle x|y\rangle= \frac{1}{2}\cdot(\frac{-i}{2})+(\frac{-1}{2})\cdot\frac{i}{2}+\frac{1+i}{2}\cdot\frac{1+i}{2}= \frac{-i}{4}-\frac{i}{4}+\frac{2i}{4}+\frac{i^2}{4}= \frac{-2i}{4}+\frac{2i}{4}+\frac{1}{4}-\frac{1}{4}=0$\\
$\Rightarrow |x\rangle$ und $|y\rangle$ sind orthogonal.\\\\

$\langle w|y\rangle= \frac{1}{3}\cdot(\frac{-i}{2})+(\frac{-2i}{3})\cdot\frac{i}{2}+\frac{2i}{3}\cdot\frac{1+i}{2}=\frac{-i}{6}+\frac{2}{6}+\frac{2i}{6}-\frac{2}{6}=\frac{i}{6}\neq 0$\\
$\Rightarrow |w\rangle$ und $|y\rangle$ sind nicht orthogonal, daher bilden $|x\rangle$, $|y\rangle$ und $|w\rangle$ kein Orthonormalsystem.

%b)
\item $\langle x|z\rangle=0 \wedge \langle y|z\rangle=0$ mit $|x\rangle = a\cdot|\spadesuit\rangle+b\cdot|\bigstar\rangle+c\cdot|\blacklozenge\rangle$\\
$\Rightarrow \frac{1}{2}\cdot a-\frac{1}{2}\cdot b+\frac{1+i}{2}c = \frac{i}{2}\cdot a- \frac{i}{2}\cdot b+\frac{i-1}{2}\cdot c$\\
$\Leftrightarrow \frac{1-i}{2} \cdot a- \frac{1-i}{2} \cdot b + i \cdot c =0$\\
Lösung für diese Gleichung: $a=b \wedge c=0$\\
Da $|z\rangle$ ebenfalls ein Einheitsvektor sein muss, muss gelten:\\
$\langle z|z\rangle=a^*\cdot a+ b^* \cdot b + c^* \cdot c =1$, wobei $a^*, b^*, c^*$ jeweils für das komplex konjugierte steht\\
Wenn nun $a$ und $b$ aus $\mathbb{R}$ gewählt werden, gilt:\\
$a\cdot a+ b \cdot b =1 \Leftrightarrow a^2+b^2= 1 \Leftrightarrow 2a^2=1 \Leftrightarrow a =\frac{1}{\sqrt{2}}$\\
$\Rightarrow |z\rangle= \frac{1}{\sqrt{2}}|\spadesuit\rangle+\frac{1}{\sqrt{2}}|\bigstar\rangle$

%c)
\item $|\frac{i}{2}|^2= (\frac{1}{2}\cdot 1)^2=\frac{1}{4}=25\% \Rightarrow$ $|\blacklozenge\rangle$ wird mit einer Wahrscheinlichkeit von $25\%$ gemessen.

%d)
\item \begin{enumerate}[i.]
\item $|x\rangle= \frac{1}{2}|\spadesuit\rangle-\frac{1}{2}|\bigstar\rangle+\frac{1-i}{2}|\blacklozenge\rangle$
\item $|y\rangle= \frac{-i}{2}|\spadesuit\rangle+\frac{i}{2}|\bigstar\rangle+\frac{1+i}{2}|\blacklozenge\rangle$
\item $|z\rangle= \frac{1}{\sqrt{2}}|\spadesuit\rangle+\frac{1}{\sqrt{2}}|\bigstar\rangle \Leftrightarrow |\spadesuit\rangle = \sqrt{2}|z\rangle-|\bigstar\rangle$
\end{enumerate}

$\Rightarrow i \cdot |x\rangle = \frac{i}{2}|\spadesuit\rangle-\frac{i}{2}|\bigstar\rangle+\frac{1+i}{2}|\blacklozenge\rangle$\\
$\Leftrightarrow i \cdot |x\rangle -|y\rangle = i|\spadesuit\rangle-i|\bigstar\rangle$ Nun wird Gleichung iii. eingesetzt.\\
$\Leftrightarrow i \cdot |x\rangle -|y\rangle = i\cdot \sqrt{2}\cdot|x\rangle-i\cdot|\bigstar\rangle-i\cdot|\bigstar\rangle= i\cdot \sqrt{2}\cdot|x\rangle-i\cdot 2\cdot |\bigstar\rangle$\\
$\Leftrightarrow 2\cdot i\cdot|\bigstar\rangle= -i \cdot|x\rangle+|y\rangle+i\cdot \sqrt{2} \cdot |z\rangle$\\
$\Leftrightarrow |\bigstar\rangle= \frac{-1}{2}|x\rangle-\frac{i}{2}|y\rangle+\frac{1}{\sqrt{2}}\cdot |z\rangle \Rightarrow$\\
Warhscheinlichkeit dass $|y\rangle$ gemessen wird: $|\frac{-i}{2}|^2=\frac{1}{4}=25\%$

\end{enumerate}
\newpage
\section*{Aufgabe 2}
Eine Matrix U ist unitär $\Leftrightarrow$ $U\cdot U^{\dagger}=E$, wobei $E$ die Einheitsmatrix ist.\\\\

\noindent$M_{\neg}\cdot M_{\neg}^{\dagger}=\begin{pmatrix} 0 & 1\\ 1 & 0\end{pmatrix}\cdot \begin{pmatrix} 0 & 1\\ 1 & 0\end{pmatrix}= \begin{pmatrix} 1 & 0\\ 0 & 1\end{pmatrix}=E_2$\\\\

\noindent$\sqrt{M_{\neg}}\cdot\sqrt{M_\neg}^\dagger=\begin{pmatrix} \frac{1+i}{2} & \frac{1-i}{2}\\ \frac{1-i}{2} & \frac{1+i}{2}\end{pmatrix}\cdot \begin{pmatrix} \frac{1-i}{2} & \frac{1+i}{2}\\ \frac{1+i}{2} & \frac{1-i}{2}\end{pmatrix}= \begin{pmatrix} \frac{1}{2}+\frac{1}{2} & 1+2i-1+1-2i-1\\ 1+2i-1+1-2i-1 & \frac{1}{2}+\frac{1}{2}\end{pmatrix}=\begin{pmatrix} 1 & 0 \\ 0 & 1\end{pmatrix} =E_2$\\\\

\noindent$\sqrt{M_\neg}^2= \begin{pmatrix}\frac{1+i}{2} & \frac{1-i}{2} \\ \frac{1-i}{2} & \frac{1+i}{2}\end{pmatrix}\cdot \begin{pmatrix}\frac{1+i}{2} & \frac{1-i}{2} \\ \frac{1-i}{2} & \frac{1+i}{2}\end{pmatrix}= \begin{pmatrix}2i-2i & \frac{1}{2}+ \frac{1}{2} \\ \frac{1}{2}+ \frac{1}{2} & 2i-2i \end{pmatrix}= \begin{pmatrix} 0 & 1 \\ 1 & 0\end{pmatrix}= M_\neg$

\newpage
\section*{Aufgabe 3}
\begin{enumerate}[a)]

%a)
\item $|v_{zwischen}\rangle=\sqrt{M_\neg}\cdot |v_{init}\rangle$\\
$|v_{final}\rangle=\sqrt{M_\neg}\cdot |v_{zwischen}\rangle= \sqrt{M_\neg}^2\cdot |v_{init}\rangle$ Es gilt: $\sqrt{M_\neg}^2=M_\neg$\\
$\Rightarrow |v_{final}\rangle=M_\neg\cdot |v_{init}\rangle= \begin{pmatrix}0 & 1\\1&0\end{pmatrix}\cdot |0\rangle= \begin{pmatrix}0\\1\end{pmatrix}= 0\cdot |0\rangle+1\cdot |1\rangle$\\
$\Rightarrow$ Ws dass $|0\rangle$ gemessen wird: $|0|^2=0\% \Rightarrow$ Ws dass $|1\rangle$ gemessen wird: $100\%$

%b)
\item $|v_{zwischen}\rangle=\sqrt{M_\neg}\cdot |v_{init}\rangle=\begin{pmatrix}\frac{1+i}{2}\\\frac{1-i}{2}\end{pmatrix}=\frac{1+i}{2}\cdot |0\rangle+\frac{1-i}{2}\cdot |1\rangle$\\
$\Rightarrow$ Ws dass $|0\rangle$ gemessen wird: $|\frac{1+i}{2}|^2=\frac{1}{2}=50\%$, Ws dass $|1\rangle$ gemessen wird: $|\frac{1-i}{2}|^2=\frac{1}{2}=50\%$\\\\
Fall 1: Es wurde $|0\rangle$ gemessen (mit Ws 50\%)
$\Rightarrow |v_{zwischen}\rangle=|0\rangle$\\
$\Rightarrow |v_{final'}\rangle=\sqrt{M_\neg}\cdot |0\rangle= \begin{pmatrix}\frac{1+i}{2} & \frac{1-i}{2}\\\frac{1-i}{2}&\frac{1+i}{2}\end{pmatrix}\cdot |0\rangle= \begin{pmatrix}\frac{1+i}{2}\\\frac{1-i}{2}\end{pmatrix}=\frac{1+i}{2}\cdot |0\rangle+\frac{1-i}{2}\cdot |1\rangle$\\
$\Rightarrow$ Ws dass $|0\rangle$ gemessen wird: $\frac{1}{2}\cdot|\frac{1+i}{2}|^2=\frac{1}{4}=25\%$, Ws dass $|1\rangle$ gemessen wird: $\frac{1}{2}\cdot|\frac{1-i}{2}|^2=\frac{1}{4}=25\%$\\\\
Fall 2: Es wurde $|1\rangle$ gemessen (mit Ws 50\%)
$\Rightarrow |v_{zwischen}\rangle=|1\rangle$\\
$\Rightarrow |v_{final'}\rangle=\sqrt{M_\neg}\cdot |1\rangle= \begin{pmatrix}\frac{1+i}{2} & \frac{1-i}{2}\\\frac{1-i}{2}&\frac{1+i}{2}\end{pmatrix}\cdot |1\rangle= \begin{pmatrix}\frac{1-i}{2}\\\frac{1+i}{2}\end{pmatrix}=\frac{1-i}{2}\cdot |0\rangle+\frac{1+i}{2}\cdot |1\rangle$\\
$\Rightarrow$ Ws dass $|0\rangle$ gemessen wird: $\frac{1}{2}\cdot|\frac{1-i}{2}|^2=\frac{1}{4}=25\%$, Ws dass $|1\rangle$ gemessen wird: $\frac{1}{2}\cdot|\frac{1+i}{2}|^2=\frac{1}{4}=25\%$\\\\
$\Rightarrow$ Insgesamt: Ws dass $|0\rangle$ gemessen wird: $2\cdot\frac{1}{4}=50\%$, Ws dass $|1\rangle$ gemessen wird: $2\cdot\frac{1}{4}=50\%$

\end{enumerate}

\end{document}
