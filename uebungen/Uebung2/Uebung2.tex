\documentclass[a4paper]{scrartcl}
\usepackage{scrpage2}
\usepackage[ngerman]{babel}
\usepackage[T1]{fontenc}
\usepackage[utf8]{inputenc}
%\usepackage[pdftex]{graphicx}
%\usepackage[intlimits]{amsmath}
%\usepackage{listings}
%\lstset{frame=single,breaklines=true}
\usepackage{ amssymb }
\usepackage{amsmath}
\usepackage{hyperref}
\usepackage{enumerate}
\usepackage[a4paper, total={19cm, 23cm}]{geometry}
\usepackage{stmaryrd}
\pagestyle{scrheadings}
\pagenumbering{gobble}
\ihead{Übungsblatt 2\\Nils Werner 108012219293}
\chead{\\Paul Rösler 108012225686	}
\ohead{Übungsgruppe: Mo. 16:00\\Daniel Teuchert 108012214552}
\setheadsepline{0.4pt}
\begin{document}

\section*{Aufgabe 1}
$|\alpha\rangle = \frac{1}{\sqrt{2}}|0\rangle + \frac{1}{\sqrt{2}}|1\rangle \land |\beta\rangle = \frac{i}{\sqrt{2}}|0\rangle - \frac{i}{\sqrt{2}}|1\rangle$\\
$\Leftrightarrow |1\rangle = \sqrt{2} |\alpha\rangle-|0\rangle \land |0\rangle = \frac{\sqrt{2}}{i} |\beta\rangle+|1\rangle = \frac{\sqrt{2}}{i} |\beta\rangle + \sqrt{2} |\alpha\rangle-|0\rangle$\\
$\Leftrightarrow |1\rangle = \sqrt{2} |\alpha\rangle-\frac{\sqrt{2}}{i} |\beta\rangle-|1\rangle \land |0\rangle = \frac{\sqrt{2}}{2i} |\beta\rangle + \frac{1}{\sqrt{2}} |\alpha\rangle$\\
$\Leftrightarrow |1\rangle = \frac{1}{\sqrt{2}} |\alpha\rangle+\frac{i}{\sqrt{2}} |\beta\rangle \land |0\rangle =\frac{1}{\sqrt{2}} |\alpha\rangle - \frac{i}{\sqrt{2}} |\beta\rangle $\\
$\Rightarrow |z\rangle = \frac{1+i}{2} |0_1 0_2 0_3\rangle  + \frac{1}{2} |1_1 1_2 0_3\rangle + \frac{i}{2} |1_1 0_2 1_3\rangle $\\
$= \frac{1+i}{2}( \frac{1}{\sqrt{2}} |\alpha_1\rangle - \frac{i}{\sqrt{2}} |\beta_1\rangle \bigotimes \frac{1}{\sqrt{2}} |\alpha_2\rangle - \frac{i}{\sqrt{2}} |\beta_2\rangle \bigotimes \frac{1}{\sqrt{2}} |\alpha_3\rangle - \frac{i}{\sqrt{2}} |\beta_3\rangle )$\\
$+ \frac{1}{2} (\frac{1}{\sqrt{2}} |\alpha_1\rangle+\frac{i}{\sqrt{2}} |\beta_1\rangle \bigotimes \frac{1}{\sqrt{2}} |\alpha_2\rangle+\frac{i}{\sqrt{2}} |\beta_2\rangle \bigotimes \frac{1}{\sqrt{2}} |\alpha_3\rangle - \frac{i}{\sqrt{2}} |\beta_3\rangle ) $\\
$+ \frac{i}{2} (\frac{1}{\sqrt{2}} |\alpha_1\rangle+\frac{i}{\sqrt{2}} |\beta_1\rangle \bigotimes \frac{1}{\sqrt{2}} |\alpha_2\rangle - \frac{i}{\sqrt{2}} |\beta_2\rangle \bigotimes \frac{1}{\sqrt{2}} |\alpha_3\rangle+\frac{i}{\sqrt{2}} |\beta_3\rangle ) $\\
$= \frac{1+i}{\sqrt{2}} |\alpha_1\rangle \bigotimes (\frac{1+i}{\sqrt{2}} |\alpha_2\rangle + \frac{1}{\sqrt{2}} |\beta_2\rangle) \bigotimes \frac{1+i}{\sqrt{2}} |\alpha_3\rangle $\\

Wsk, bei $|z\rangle$ im 1. Qubit $|\alpha\rangle$ zu messen: $|\frac{1+i}{\sqrt{2}}| = 1$\\
Zustand danach: $\frac{1+i}{\sqrt{2}} |\alpha_1\rangle \bigotimes (\frac{1+i}{\sqrt{2}} |\alpha_2\rangle + \frac{1}{\sqrt{2}} |\beta_2\rangle) \bigotimes \frac{1+i}{\sqrt{2}} |\alpha_3\rangle $\\

Wsk, bei $|z\rangle$ im 1. Qubit $|\beta\rangle$ zu messen: $0$\\
Zustand danach kann nicht angegeben werden, da das Ereignis nicht eintritt.\\

Wsk, bei $|z\rangle$ im 3. Qubit $|\alpha\rangle$ zu messen: $|\frac{1+i}{\sqrt{2}}| = 1$\\
Zustand danach: $\frac{1+i}{\sqrt{2}} |\alpha_1\rangle \bigotimes (\frac{1+i}{\sqrt{2}} |\alpha_2\rangle + \frac{1}{\sqrt{2}} |\beta_2\rangle) \bigotimes \frac{1+i}{\sqrt{2}} |\alpha_3\rangle $\\

Wsk, bei $|z\rangle$ im 3. Qubit $|\beta\rangle$ zu messen: $0$\\
Zustand danach kann nicht angegeben werden, da das Ereignis nicht eintritt.\\
\newpage
\section*{Aufgabe 2}
$P_{n,n} = 
 \begin{pmatrix}
  p_{1,1} & p_{1,2} & \cdots & p_{1,n} \\
  p_{2,1} & p_{2,2} & \cdots & p_{2,n} \\
  \vdots  & \vdots  & \ddots & \vdots  \\
  p_{n,1} & p_{n,2} & \cdots & p_{n,n} 
 \end{pmatrix}$ mit $
P_{n,n}^{\dagger} = 
 \begin{pmatrix}
  p_{1,1} & p_{2,1} & \cdots & p_{n,1} \\
  p_{1,2} & p_{2,2} & \cdots & p_{n,2} \\
  \vdots  & \vdots  & \ddots & \vdots  \\
  p_{1,n} & p_{2,n} & \cdots & p_{n,n} 
\end{pmatrix}$\\\\
Z.z: $P_{n,n}^{\dagger} = P_{n,n}^{-1} \Leftrightarrow P_{n,n} \cdot P_{n,n}^{\dagger} = E_n$:\\
$\begin{pmatrix}
p_{1,1} & p_{1,2} & \cdots & p_{1,n} \\
p_{2,1} & p_{2,2} & \cdots & p_{2,n} \\
\vdots  & \vdots  & \ddots & \vdots  \\
p_{n,1} & p_{n,2} & \cdots & p_{n,n} 
\end{pmatrix} \cdot
\begin{pmatrix}
p_{1,1} & p_{2,1} & \cdots & p_{n,1} \\
p_{1,2} & p_{2,2} & \cdots & p_{n,2} \\
\vdots  & \vdots  & \ddots & \vdots  \\
p_{1,n} & p_{2,n} & \cdots & p_{n,n} 
\end{pmatrix}$\\
$ =
\begin{pmatrix}
p_{1,1}^2+p_{1,2}^2+...+p_{1,n}^2 & p_{1,1}p_{2,1}+p_{1,2}p_{2,2}+...+p_{1,n}p_{2,n} & \cdots & p_{1,1}p_{n,1}+p_{1,2}p_{n,2}+...+p_{1,n}p_{n,n} \\
p_{2,1}p_{1,1}+p_{2,2}p_{1,2}+...+p_{2,n}p_{1,n} & p_{2,1}^2+p_{2,2}^2+...+p_{2,n}^2 & \cdots & p_{2,1}p_{n,1}+p_{2,2}p_{n,2}+...+p_{2,n}p_{n,n} \\
\vdots  & \vdots  & \ddots & \vdots  \\
p_{n,1}p_{1,1}+p_{n,2}p_{1,2}+...+p_{n,n}p_{1,n} & p_{n,1}p_{2,1}+p_{n,2}p_{2,2}+...+p_{n,n}p_{2,n} & \cdots & p_{n,1}^2+p_{n,2}^2+...+p_{n,n}^2 
\end{pmatrix}$\\\\
$p_{i,1}^2+p_{i,2}^2+...+p_{i,n}^2=1~\forall i \in \{1,...,n\}$ da hier jeweils die Quadrate aller Einträge einer Zeile summiert werden. Da in einer Zeile nur an einer Stelle eine 1 steht, ergibt das Quadrat nur an dieser Stelle eine 1, sonst eine 0.\\
$p_{i,1}p_{j,1}+p_{i,2}p_{j,2}+...+p_{i,n}p_{j,n} = 0~\forall i\neq j \in \{1,...,n\}$ da hier jeweils zwei Einträge aus einer Spalte miteinander multipliziert werden, die nicht in der gleichen Zeile stehen. Da die Faktoren ungleich sind und damit einer der beiden 0 sein muss, ist das Produkt jeweils 0. Die Summe über diese Produkte ist damit 0.\\
$\Rightarrow P_{n,n} \cdot P_{n,n}^{\dagger} =
\begin{pmatrix}
1 & 0 & \cdots & 0 \\
0 & 1 & \cdots & 0 \\
\vdots  & \vdots  & \ddots & \vdots  \\
0 & 0 & \cdots & 1 
\end{pmatrix} = E_n$\\
$\Rightarrow P_{n,n}$ ist eine unitäre Matrix.

\newpage
\section*{Aufgabe 3}
$|y\rangle = |y_1 y_2...y_n\rangle = |y_1\rangle \bigotimes ... \bigotimes |y_n\rangle$\\
$H_n|y\rangle =\bigotimes_{i=1}^nW_2|y\rangle =(W_2\bigotimes W_2\bigotimes ... \bigotimes W_2)(|y_1\rangle \bigotimes |y_2\rangle \bigotimes ... \bigotimes |y_n\rangle) = (W_2|y_1)\bigotimes (W_2|y_2\rangle)\bigotimes ... \bigotimes (W_2|y_n\rangle)$\\
Es gilt: $W_2|0\rangle = \frac{1}{\sqrt{2}}(|0\rangle + |1\rangle)$ und $W_2|1\rangle = \frac{1}{\sqrt{2}}(|0\rangle - |1\rangle)$\\
$\Rightarrow (\frac{1}{\sqrt{2}}(|0\rangle + (-1)^{y_1}|1\rangle ))\bigotimes (\frac{1}{\sqrt{2}}(|0\rangle + (-1)^{y_2} |1\rangle)) \bigotimes ... \bigotimes (\frac{1}{\sqrt{2}}(|0\rangle + (-1)^{y_n}|1\rangle))$\\
$=  2^{-\frac{n}{2}}(|00...0\rangle + (-1)^{y_n}|00...01\rangle+ (-1)^{y_{n-1}}|00...010\rangle + (-1)^{y_n+y_{n-1}}|00....011\rangle + ... + (-1)^{\sum_{i=1}^n y_i}|11...1\rangle$\\
$=  2^{-\frac{n}{2}}((-1)^{|0...0\rangle \cdot y }|00...0\rangle + (-1)^{|00...01\rangle \cdot y}|00...01\rangle+ (-1)^{|00...010\rangle \cdot y}|00...010\rangle + (-1)^{|00...011\rangle \cdot y}|00...011\rangle + ... + (-1)^{|11...1\rangle \cdot y}|11...1\rangle$\\
$= 2^\frac{n}{2}\sum_{x = 0^n \in \{0,1\}^n}^{1^n}(-1)^{x\cdot y}|x\rangle = H_n|y\rangle$

\newpage
\section*{Aufgabe 4}
\begin{enumerate}[a)]

%a)
\item Es muss gelten: $U |00\rangle=\frac{1}{\sqrt{2}}(|00\rangle + |11\rangle )=|\beta_{00}\rangle$\\
Des weiteren gelte: $U |01\rangle=\frac{1}{\sqrt{2}}(|01\rangle + |10\rangle )=|\beta_{01}\rangle$, $U |10\rangle=\frac{1}{\sqrt{2}}(|00\rangle - |11\rangle )=|\beta_{10}\rangle$, $U |11\rangle=\frac{1}{\sqrt{2}}(|01\rangle - |10\rangle )=|\beta_{11}\rangle$\\
$\Rightarrow U = |\beta_{00}\rangle \langle 00| + |\beta_{01}\rangle \langle 01| + |\beta_{10}\rangle \langle 10| + |\beta_{11}\rangle \langle 11|$\\
$= \frac{1}{\sqrt{2}} \Bigg( \begin{pmatrix}
1&0&0&0\\
0&0&0&0\\
0&0&0&0\\
1&0&0&0
\end{pmatrix} + \begin{pmatrix}
0&0&0&0\\
0&1&0&0\\
0&1&0&0\\
0&0&0&0
\end{pmatrix} + \begin{pmatrix}
0&0&1&0\\
0&0&0&0\\
0&0&0&0\\
0&0&-1&0
\end{pmatrix} + \begin{pmatrix}
0&0&0&0\\
0&0&0&1\\
0&0&0&-1\\
0&0&0&0
\end{pmatrix}\Bigg)$\\
$=\frac{1}{\sqrt{2}} \begin{pmatrix}
1&0&1&0\\
0&1&0&1\\
0&1&0&-1\\
1&0&-1&0
\end{pmatrix}$\\
Z.z: $U^{\dagger} = U^{-1} \Leftrightarrow U \cdot U^{\dagger} = E_4$:\\
$\frac{1}{\sqrt{2}} \begin{pmatrix}
1&0&1&0\\
0&1&0&1\\
0&1&0&-1\\
1&0&-1&0
\end{pmatrix} \cdot \frac{1}{\sqrt{2}} \begin{pmatrix}
1&0&0&1\\
0&1&1&0\\
1&0&0&-1\\
0&1&-1&0
\end{pmatrix} = \frac{1}{2} \begin{pmatrix}
2&0&0&0\\
0&2&0&0\\
0&0&2&0\\
0&0&0&2
\end{pmatrix} = E_4$

%b)
\item Nein, da:\\
$(U' \bigotimes U') |00\rangle = |ERP\rangle$\\
$\Leftrightarrow (U' |0_1\rangle) \bigotimes (U' |0_2\rangle) = |ERP\rangle$\\
$\Leftrightarrow |ERP_1\rangle \bigotimes |ERP_2\rangle = |ERP\rangle~ \lightning$\\
Da $|ERP\rangle$ verschränkt ist, kann es nicht als Tensorprodukt zweier Vektoren berechnet werden.
\end{enumerate}
\newpage
\section*{Aufgabe 5}
\begin{enumerate}[a)]

%a)
\item
\end{enumerate}
\end{document}
