\documentclass[a4paper]{scrartcl}
\usepackage{scrpage2}
\usepackage[ngerman]{babel}
\usepackage[T1]{fontenc}
\usepackage[utf8]{inputenc}
%\usepackage[pdftex]{graphicx}
%\usepackage[intlimits]{amsmath}
%\usepackage{listings}
%\lstset{frame=single,breaklines=true}
\usepackage{ amssymb }
\usepackage{amsmath}
\usepackage{hyperref}
\usepackage{enumerate}
\pagestyle{scrheadings}
\pagenumbering{gobble}
\ihead{Übungsblatt 2\\Nils Werner 108012219293}
\chead{\\Paul Rösler 108012225686	}
\ohead{Übungsgruppe: Mo. 16:00\\Daniel Teuchert 108012214552}
\setheadsepline{0.4pt}
\begin{document}

\section*{Aufgabe 1}
Wsk, bei $|z\rangle$ im 1. Qubit $|\alpha\rangle$ zu messen:\\

Zustand danach:\\
Wsk, bei $|z\rangle$ im 1. Qubit $|\beta\rangle$ zu messen:\\

Zustand danach:\\
Wsk, bei $|z\rangle$ im 3. Qubit $|\alpha\rangle$ zu messen:\\

Zustand danach:\\
Wsk, bei $|z\rangle$ im 3. Qubit $|\beta\rangle$ zu messen:\\

Zustand danach:\\
\newpage
\section*{Aufgabe 2}


\newpage
\section*{Aufgabe 3}


\newpage
\section*{Aufgabe 4}
\begin{enumerate}[a)]

%a)
\item Es muss gelten: $U |00\rangle=\frac{1}{\sqrt{2}}(|00\rangle + |11\rangle )$\\
$\Leftrightarrow U = \frac{1}{\sqrt{2}}(|00 \rangle + |11\rangle)\cdot \langle 00|= \frac{1}{\sqrt{2}}(|00\rangle \langle 00|+|11\rangle \langle 00|)= \begin{pmatrix}
1&0&0&0\\
0&0&0&0\\
0&0&0&0\\
1&0&0&0
\end{pmatrix}$
\end{enumerate}
\newpage
\section*{Aufgabe 5}
\begin{enumerate}[a)]

%a)
\item
\end{enumerate}
\end{document}
