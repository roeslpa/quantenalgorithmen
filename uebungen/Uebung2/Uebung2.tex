\documentclass[a4paper]{scrartcl}
\usepackage{scrpage2}
\usepackage[ngerman]{babel}
\usepackage[T1]{fontenc}
\usepackage[utf8]{inputenc}
%\usepackage[pdftex]{graphicx}
%\usepackage[intlimits]{amsmath}
%\usepackage{listings}
%\lstset{frame=single,breaklines=true}
\usepackage{ amssymb }
\usepackage{amsmath}
\usepackage{hyperref}
\usepackage{enumerate}
\pagestyle{scrheadings}
\pagenumbering{gobble}
\ihead{Übungsblatt 2\\Nils Werner 108012219293}
\chead{\\Paul Rösler 108012225686	}
\ohead{Übungsgruppe: Mo. 16:00\\Daniel Teuchert 108012214552}
\setheadsepline{0.4pt}
\begin{document}

\section*{Aufgabe 1}
Wsk, bei $|z\rangle$ im 1. Qubit $|\alpha\rangle$ zu messen:\\
$|(|\alpha\rangle\langle\alpha|\otimes id)|z\rangle|^2= ((\frac{1}{\sqrt{2}}|0\rangle+\frac{1}{\sqrt{2}}|1\rangle)(\frac{1}{\sqrt{2}}\langle 0|+\frac{1}{\sqrt{2}}\langle 1|)\otimes id )\frac{1+i}{2}|000\rangle+\frac{1}{2}|110\rangle+\frac{i}{2}|101\rangle$\\
$= \frac{1}{2}|0\rangle\langle 0|+$\\
Zustand danach:\\
Wsk, bei $|z\rangle$ im 1. Qubit $|\beta\rangle$ zu messen:\\
$|(|\beta\rangle\langle\beta|\otimes id)|z\rangle|^2= $\\
Zustand danach:\\
Wsk, bei $|z\rangle$ im 3. Qubit $|\alpha\rangle$ zu messen:\\
$|(|id\otimes\alpha\rangle\langle\alpha|)|z\rangle|^2= $\\
Zustand danach:\\
Wsk, bei $|z\rangle$ im 3. Qubit $|\beta\rangle$ zu messen:\\
$|(|id\otimes\beta\rangle\langle\beta|)|z\rangle|^2= $\\\\
Zustand danach:\\
\newpage
\section*{Aufgabe 2}


\newpage
\section*{Aufgabe 3}

\newpage
\section*{Aufgabe 4}
\begin{enumerate}[a)]

%a)
\item

\end{enumerate}
\newpage
\section*{Aufgabe 5}
\begin{enumerate}[a)]

%a)
\item
\end{enumerate}
\end{document}
